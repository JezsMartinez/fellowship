\documentclass[notitlepage,12pt]{report}
\usepackage[left=0.5in, right=0.5in, top=0.5in, bottom=0.5in]{geometry}

\usepackage{titling}
\usepackage{lipsum}
\usepackage{braket}
\usepackage{graphicx}
\usepackage[table]{xcolor}
\graphicspath{./images/}
\usepackage{subcaption}
\newcommand{\tr}{\mathrm{tr}}
\usepackage{hyperref}
\usepackage{authblk}
\usepackage[backend=biber, style=chem-acs]{biblatex}
\usepackage{tabularx}
% \usepackage{amsmath}
\usepackage{mathtools}% Loads amsmath
\usepackage{physics}
\renewcommand\thesection{\arabic{section}}

\bibliography{bib}
\addbibresource{bib.bib}


\begin{document}
	\title{Electronic properties in condensed-phase molecular systems under Embedded theoretical approaches: Liquid water systems}
	\author[1]{Jessica Martinez}
	\affil[1]{Department of Chemistry, Rutgers University, Newark, New Jersey}
	\date{June 2022}
	\renewcommand\Affilfont{\itshape\small}
	\thispagestyle{empty}
\maketitle
\section{Introduction}

	High accurate description at the electronic level in condensed-phase molecular systems as liquid water requires a perfect equilibrium between the use of spectroscopy techniques\supercite{reimann2021two,malerz2021low,bolognesi2021combined} and computational approaches \supercite{couto2007understanding,ambrosio2016structural,ozaki2021advances}. Two important properties widely used in liquids characterization imply the determination of Ionization Potentials (IPs)\supercite{thurmer2021accurate,perry2020ionization,credidio2021quantitative,thurmer2021valence,tolle2019charged,gaiduk2018electron,gaiduk2016photoelectron,seidel2016valence}, as well as, the core electronic excitations to the continuous by the X-Ray absorption spectroscopy (XAS)\supercite{zhovtobriukh2019liquid,zhang2020isotope,smith2020femtosecond}.
	
	The determination of IPs of liquid water take part in many crucial processes in electrochemistry\supercite{marenich2014computational}, photochemistry\supercite{reuther1996primary,hu2021photochemical} as well as peculiar states of matter like excess electrons solvated in liquid water\supercite{ambrosio2017electronic}.  Recent advances in liquid microjet (LJ) photoelectron spectroscopy have opened the door to the determination of accurate electronic energetics of water and aqueous solutions \supercite{thurmer2021accurate,perry2020ionization,credidio2021quantitative,thurmer2021valence}. The most accurate value for the VIE of liquid water employing LJ is 11.33 $\pm$0.03 eV which account for a critical factor: the measuring of the liquid-phase low-energy cutoff spectrum along with the photoelectron peak of the interest\supercite{thurmer2021accurate}. 
	
	Theoretical studies as Gaiduk \textit{et al.} \supercite{gaiduk2018electron} employed path-integral molecular dynamics (PIMD) in combination with many-body perturbation theory (MBPT) within the $G_{0}W_{0}$ approximation starting from a range-separate hybrid (RSH) in the determination of the IP, with a VIE of 11.53 eV. Similarly, Ziaei \textit{et al}. \supercite{ziaei2018probing} utilized a self-consistent GW approach with a vertex correction and the projector augmented wave (PAW) method \supercite{dal2014pseudopotentials} finding  good agreement with the results of Gaiduk \textit{et al.} \supercite{gaiduk2018electron}. Both results are remarkable in that they reproduce the experiment to within a few tenths of eVs.
	
	Following the trend but now employing Density Functional Theory (DFT) \supercite{thomas1927calculation,fermi1927statistical, hogenberg, kohn} a previous work\supercite{tolle2019charged} attempted the calculation of IPs of liquid water based on a new subsystem DFT method\supercite{jacob2014subsystem,wesolowski2015frozen,krishtal2015subsystem} capable of approaching charged systems regardless of the underlying adoption of periodic boundary conditions \supercite{tolle2019charged}, called “Impurity Model”. 
	
	Subsystem DFT ameliorates the limitation of DFT in the studying of bulk systems as liquids, based on a divide-and-conquer strategy. In which the total electron density is divided into a sum of individual densities called subsystem DFT or Frozen Density Embedding (FDE) \supercite{mi2021eqe,mi2019nonlocal,mi2019ab}. By partitioning a bulk-phase system into smaller subsystems allows the selection of a few or even one subsystem of interest, which as a consequence is embedded in a complex environment \supercite{schmitt2020frozen}. This arises to a robust strategy to study the effects of the environment due to a single subsystem, the so-called “Embedding Potential” \supercite{genova2016avoiding}. 
	
	The “Impurity Model” relies on the fact that Coulomb potentials for the electronic and nuclear charges can be computed based on a Screening potential arising for the contribution of the “Embedding potential” of the periodic images with respect to the same subsystem\cite{tolle2019charged}.
	
	Based on both the “Impurity Model” and adding the $\Delta$-SCF approach\supercite{bagus1965self,waskom2017mwaskom} (the energy difference between the neutral and the polarized systems) not only the vertical IP of liquid water are computed, but also the IP is breakdown into five energy contributions:  (1) Mean field: the ionization energy of each water molecule computed at Hartree-Fock level; (2) Correlation: electron correlation within each water molecule at the correlated wavefunction level as well as DFT; (3) Interaction: Coulomb interactions between each water molecule and their environment augmented by effects of exchange, correlation and Pauli repulsion; (4) Polarization: polarization of the environment electronic structure in response to the ionization of a nearby water molecule; and finally, (5) Delocalization: the possibility that the spin density of the cation is delocalized over more than one water molecule.
	
	By averaging over 128 multiple snapshots of 64 water molecules considered in the corresponding simulation cell two \textit{ab-initio} molecular dynamics (AIMD) trajectories were used separately, a Born-Oppenheimer(BO) and a path-integral(PI) dynamics trajectory \supercite{gaiduk2018electron}. The average mean-field contribution is then successfully obtained (av. 10.81 eV) and reformed by three essential corrections: Correlation, Interaction, and Polarization. 
	
\begin{minipage}{0.4\textwidth}
	\label{cont_ip}
	\includegraphics[width=\linewidth]{./images/contribution_liquidwater_PI}
	{Contribution chart for the first 10 snapshots form PI dynamics trajectory. \supercite{gaiduk2018electron}}
\end{minipage}
\begin{minipage}{0.55\textwidth}
	In terms of the correlation, we found a consistent trend among CCSD(T), PBE, and MP2 methods, see Figure \ref{cont_ip}, with a slight overestimation of av. 0.28 eV for MP2, compared with one of the most accurate treatments of the electronic correlation, CCSD(T). The polarization of the electronic environment nearby an ionized water molecule arises with a correction average of av. 0.84 eV while the embedding correction subtracts av. -2.80 eV for both PI and BO systems. Subsequently, a fifth contribution was found to have a negligible effect, namely, accounting for spin density delocalization over more than one water molecule. 
\end{minipage}
	
	Going beyond FDE, an alternative embedding approach connecting subsystem DFT emerged, the so-called block-orthogonalized Manby-Miller embedding (BOMME). BOMME is a projection operator technique that is forgoing the use of approximate Kinertic Energy Densigy Functionals (KEDFs) and allows the fragmentation of a system through covalent bonds\supercite{ding2017embedded}. BOMME with a combination of real-time TDDFT (rt-BOMME) has presented a better performance in the capture of both intermolecular and intramolecular coupling among subsystems \supercite{koh2017accelerating}.
	
	X-ray absorption near-edge structures (XANES) also plays an important role in the characterization of the core electronic excitations. XANES targets the study of excitation from core orbitals due to absorbing photons in the X-ray energy range\supercite{fransson2016x}, which allow the study of the oxidation state, local symmetry, and coordination environment in gas, liquid, or solid phase\supercite{rehr2005progress,koningsberger1987x}. 
	Particularly, for the simulation of XAS in liquids multiple theoretical approaches have came up, Transition potential(TP)-DFT \supercite{triguero1998calculations}, TDDFT employing Davidson procedure\supercite{davidson197514}, Excited-state core hole (XCH)\supercite{prendergast2006x}, Bethe-Salpeter equation (BSE) and Hedin’s GW approximation of the quasiparticle \supercite{vinson2012theoretical}, Coulomb hole and screened exchange (COHSEX)\supercite{chen2010x}, and QM/MM techniques\supercite{list2014lanczos}. In summary, these approaches’ main failures range from the only qualitative prediction of the experiment (TP-DFT), as well as, under or overestimation of post or main edge energies (COHSEX,BSE). A detailed description of each approach is found in the review \cite{fransson2016x}.
	
	In this work, taking advantage of the newest implementation of rt-BOMME and rt-sDFT \supercite{de2021environment} in Psi4 code \supercite{smith2020psi4}, to predict and overcome the already mentioned failures in the modeling of the X-ray absorption near-edge structures (XANES), we wish to go further in applications of the method accounting for more than first solvation shells. To accomplish this, we selected three main clusters (1) Liquid water and Ice, (2) Cations solvated in water, and (3) A complex molecule ($K_3[Fe(CN)_6]$) solvated in water. Additionally, implementation of rt-sDFT in ADF\supercite{te2001chemistry} and eDFTpy\supercite{edftpy}, code-based of Quantum espresso package\supercite{giannozzi2009quantum}, will be used.

\section{Theoretical Background}

\subsection{Subsystem DFT}
Subsystem DFT or Frozen Density embedding formalism developed by Wesolowski and Warshel employs a divide-and-conquer approach where the Kohn--Sham (KS) problem is solved for fragments of small size, the so-called subsystems\supercite{sDFT,wesolowski1993frozen,wesolowski2006one}. The electron density of the total system $\rho$ is partitioned into subsystem electron densities and can be written as

\begin{equation}\label{eq:sumdensity}
	\rho(r) = \sum\limits_{I=1}^{ \# \ of \ subsystems} \rho_{I}(r),
\end{equation}

where the index $I$ runs over all the subsystems.

The total electronic energy becomes a function of all the subsystem densities and has the form,

\begin{equation}
	E[\{\rho_I\}] = \sum\limits_I T_{s}[\rho_I] + V_{\mathrm{nuc}}[\rho] + E_H[\rho] + E_{\mathrm{xc}}[\rho] + T_{s}^{\mathrm{nad}}[\{\rho_I\}],
\end{equation}

where $V_{\mathrm{nuc}}[\rho]=\int \rho(r) v_\mathrm{ext}(r)dr$, and $E_H[\rho]$ and $E_{\mathrm{xc}}[\rho]$ are the Hartree  and exchange-correlation energy functionals.

The variational problem is cleared up by solving self-consistently a KS-like equation per subsystem with constrained electron density \supercite{wesolowski1994ab}, which reads.

\begin{equation}\label{eq:ksced}
	\left( -\frac{\nabla^2}{2} + v_{\mathrm{eff}}^{I}[\rho_I](r) + v_{\mathrm{emb}}^{I}[\rho_{I},\rho](r) \right) \psi_{i}^I(r) = \epsilon_{i}^I \psi_{i}^I(r)
\end{equation}
The subsystem orbital energies are denoted by $\epsilon_{i}^I$ in Eq.\ \ref{eq:ksced}.\supercite{sDFT}, and the subsystem effective KS potential ($v_{\mathrm{eff}}^{I}[\rho_I](r)$) is defined as

\begin{equation}\label{equ6}
	v_{\mathrm{eff}}^{I}[\rho_I](r) = v_{\mathrm{ext}}^I(r) + v_{{H}}[\rho_I](r) + v_{\mathrm{xc}}[\rho_I](r).
\end{equation}

In equation \ref{equ6},  $v_{\mathrm{ext}}^I(r)$ is the  external potential of subsystem $I$ while $v_{\mathrm{H}}[\rho_I](r)$ and $v_{\mathrm{xc}}[\rho_I](r)$ represent the Hartree and the exchange-correlation potentials per subsystem $I$.

By way of explanation, the Kohn–Sham single-particle Hamiltonian is augmented by an embedding potential, in which  are encoded the interaction of the electrons of subsystem $I$ with the environment, which reads as follows,

\begin{equation}\label{embedd}
	\begin{aligned}
		v_{\mathrm{emb}}^{I}[\rho,\rho_I](r) = \sum\limits_{J,J \neq I} &v_{\mathrm{ext}}^{\; J}(r) + v_{{H}}[\rho - \rho_I](r)  \\ + \ &v_{\mathrm{xc}}^{\mathrm{nad}}[\rho,\rho_{I}](r) + v_{T_s}^{\mathrm{nad}}[\rho,\rho_{I}](r),
	\end{aligned}
\end{equation}
where $v_{\mathrm{ext}}^{\; J}$ and $v_{{H}}[\rho - \rho_I](r)$ are the inter-subsystem potentials that capture the Coulomb interactions. 

\subsection{rt-TDDFT}
Real-Time Time-Dependent Density Functional Theory (rt-TDDFT) or Real-Time Time-Dependent Kohn-Sham is used for computing excited state and response properties of molecules goes beyond Casida's equation \supercite{casida1995recent} and use explicit time propagation methods\supercite{goings2018real}.

Evolution in time of the one-electron density matrix $\textbf{D}(t)$ in the algebraic approximation is given by,

\begin{equation}
	\textbf{D}(t) = \textbf{U}(t,t_0)\textbf{D}(t_0)\textbf{U}(t,t_0)^{\dagger}
\end{equation}
Where $\textbf{U}(t,t_0)$, the matrix representation of the time-evolution operator, is defined as,

\begin{equation}
	\textbf{U}(t,t_0)= \hat{T} exp \left( -i \int_{t_0}^{t} \textbf{F}(t')dt' \right),
\end{equation}

and $\hat{T}$ is the time-ordering operator. This time-ordering problem is solved by discretizing the time using a short time step\supercite{de2020pyberthart}. 

A exponential midpoint ansatz is employed to study valence and core excitation \supercite{de2021environment}. Where the Fock matrix is given by,

\begin{equation}
	\textbf{F}(t)=\textbf{h}_0+\textbf{G}[\textbf{D}(t)]+v_{ext}(t)
\end{equation}

The one-electron operator is $\textbf{h}_0$ and $\textbf{G}$ is the two-electron term,  define as,

\begin{equation}
	\textbf{G}[\textbf{D}(t)] = \textbf{J}[\textbf{D}(t)]+c_x\textbf{K}[\textbf{D}(t)]+c_x\textbf{V}_xc[\textbf{D}(t)]
\end{equation}

The latter accounts for the term $c_x$, defined as the fraction of Hartree-Fock exchange in the exchange-correlation potential($V_xc$).

\subsection{BOMME}
Manby-Miller’s embedding approach accelerates rt-TDDFT employing different levels of theory into two different mapping domains. Based on the Fock-matrix(FM), the subsystem to be treated accurately is described by a high-level FM, while the remaining part with a low-level RM in a reduced basis set. A change in the basis set from the non-orthogonal atomic-orbital(AO) partition of the system to a block-orthogonalized (BO) scheme also is employed \supercite{ding2017embedded}. 

The low-level FM is defined as,

\begin{equation}
	\label{llfm}
	\tilde{\textbf{h}}_0 = \textbf{O}^{T}\textbf{h}_0\textbf{O}, \;  \tilde{G}^{Low}[\tilde{\textbf{D}}]= \textbf{O}^{T}\textbf{G}^{Low}\textbf{O}, \;  \tilde{\textbf{D}} = \textbf{O} \textbf{D} \textbf{O}^{T}
\end{equation} 

Equation \ref{llfm} describes the block-orthogonalized basis by tildes, where $\textbf{O}$ is the transformation matrix from the AO to BO basis set, defined as,

\begin{equation}
	\textbf{O}
	\begin{pmatrix}
		\textbf{I}^{AA} & \textbf{-P}^{AB} \\
		\textbf{0} & \textbf{I}^{BB}  
	\end{pmatrix}
\end{equation}

where $AA$ block denoted the subsystem with high of level theory and $BB$ block with low level of theory.$\textbf{I}^{AA}$ and $\textbf{I}^{BB}$ are the identity matrices, with dimensions $n_a$ and $n_b$, describing subsystem A and B basis set, respectively.  $\textbf{-P}^{AB}$ is the projection matrix, equal to $(\textbf{S}^{AA})^{-1}\textbf{S}^{AB}$, in which $\textbf{S}^{AB}$ is the AO overlap between the subsystems. 

Finally, the Fock matrix in the BO basis expression is,

\begin{equation}
	\textbf{F} = \tilde{h}_0 + \tilde{G}^{Low}[\tilde{\textbf{D}}] +(\tilde{\textbf{G}}^{High}[\tilde{\textbf{D}}^{AA}]-\tilde{\textbf{G}}^{Low}[\tilde{\textbf{D}^{AA}}])
\end{equation}

Different methods for the calculation of the exchange term in $\textbf{G}^{High}$ are available; here, the Koh, Nguyen-Beck, Parkhill is employed. This method accounts only for the exact exchange interaction within the AA block and reads as,

\begin{equation}
	E_{EXO}=-\frac{1}{4} \sum_{\mu kv\lambda} (\mu k|v\lambda) \textbf{D}_{\mu v}^{AA} \textbf{D}_{k \lambda}^{AA}
\end{equation}

\section{Computational Details}

Equation \ref{embedd} is solved following the frozen density embedding (FDE) method \ref{eq:ksced} implemented in the embedding Quantum Espresso (eQE) package \supercite{genova2017eqe}- Ultrasoft pseudopotentials from the PSL pseudopotential library \supercite{corso2014comput} are employed. A new implementation also is available in the package eDFTpy\supercite{edftpy} which is further used. Both codes are based on Quantum Espresso \supercite{qe}. Regarding the evaluation of the additive and nonadditive exchange--correlation contributions PBE functional \supercite{perdew1996phys} was selected. While, revAPBEK \supercite{laricchia2011generalized} for nonadditive kinetic energy contributions. 

For the rt-BOMME and rt-FDE calculation Psi4Numpy\supercite{smith2018psi4numpy} framework of Psi4 code\supercite{smith2020psi4}, is employed. B3LYP basis set employs aug-cc-pVTZ and STO-3G for the High and Low Fock Matrix calculations. For real-time simulations, the subsystem belonging to the high Fock Matrix is calculated without an external electric field and perturbed by an analytic $\delta$-function pulse os strength $k=5.0x10^{-4}$ a.u. Along with the three spatial directions. The Pade approximant-based Fourier Transform is used to furthers reduce the length of the signal for the dipole moment analysis, with a damping $\epsilon^{-\lambda \dot{t}}$ with $\lambda=3.0x10-4$. 

Three key systems are used: (1) Liquid water\supercite{gaiduk2018electron} and Ice \supercite{bergmann2007nearest,zhovtobriukh2019x}, (2) Cations ($Li^{+}$,$Na^{+}$,$K^{+}$,$NH_4^{+}$) paired to carboxylate groups of acetate and glycine solvated in water \supercite{aziz2008cation}, and (3) A complex molecule ($K_3[Fe(CN)_6]$) solvated in water \supercite{zheng2018enabling}. For system (1) the liquid water model are taken over an average structure from an \textit{ab-initio} molecular dynamics (AIMD) trajectories using path-integral(PI) approach \supercite{gaiduk2018electron}, with cell sizes of 12.42 \AA and 64 water molecules per unit cell.  Ice structures also are taken from a PIMD with 192 water molecules in the simulation box\supercite{leetmaa2010theoretical}. 

For system (2), a first MD simulation of each of the investigated ion pairs in water is taken as starting system \supercite{aziz2008cation} of 800 water molecules in a periodic cubic cell. A further \textit{ab-initio s-DFT} molecular dynamics\supercite{genova2016avoiding} based on this geometry will be performed.  For the last system (3), we will construct potassium Ferricyanide complex solvated by water, forming the first (6 water molecules) and the second (12 water molecules) solvation shells \supercite{uudsemaa2003density,seidel2011valence}. 

\section{Current project and future directions}

With the successful approximation of IP values for liquid water and their contributions, we target that the embedding potential approach can further tackle other electronic structure-based properties, such as the core electron excitations. Different methods have been employed to achieve that, but only one (BOMME) is based on subsystem partition approaches. For that, we wish to prove that rt-BOMME and rt-sDFT are capable methods to overpass failures in previous computational approaches in the calculation of XANES spectra, as too narrow absorption bands, insufficient pre-edge intensity, or insufficient postedge intensity.

But not only do we wish to get the spectra, but also we want to go further and answer different questions regarding whether simple and complex ions substantially impact the hydrogen-bond network of water outside the first solvation shell. 

\printbibliography

\end{document}
