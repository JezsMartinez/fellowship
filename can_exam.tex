\documentclass[notitlepage,12pt]{report}
\usepackage[left=1in, right=1in, top=1in, bottom=1in]{geometry}

\usepackage{titling}
\usepackage{lipsum}
\usepackage{braket}
\usepackage{graphicx}
\usepackage[table]{xcolor}
\graphicspath{./images/}
\usepackage{subcaption}
\newcommand{\tr}{\mathrm{tr}}
\usepackage{hyperref}
\usepackage{authblk}
\usepackage[backend=biber, style=chem-acs]{biblatex}
\usepackage{tabularx}
\usepackage{amsmath}
\usepackage{physics}
\renewcommand\thesection{\arabic{section}}

\bibliography{bib}
\addbibresource{bib.bib}


\begin{document}
	\title{ Non-Standard Subsystem DFT Embedded Dynamic Simulations applied to Liquids Characterization \\ ${}$ \\ \large Ph. D. Candidacy Report}
	\author[1]{Jessica Martinez}
	\affil[1]{Department of Chemistry, Rutgers University, Newark, New Jersey}
	\date{June 2022}
	\renewcommand\Affilfont{\itshape\small}
	\thispagestyle{empty}
	
\begin{titlepage}
	\maketitle
\end{titlepage}
	
\section{Introduction}
	In the last years, the high-accuracy calculations on condensed-phase molecular systems as liquids and materials have represented a challenge in the computational and theoretical field \supercite{RN43, RN46, RN48}. The development of theory target the correct description for the many-body nature of inter-molecular interactions\supercite{RN46, RN45}, and the possibility to overcome the lack of applying quantum mechanics methods to large-scale systems over long time scales\supercite{RN49}. The first quest for using quantum mechanics methods to condensed phase systems with large unit cells relies on going beyond the Schrödinger equation \supercite{schrodinger1926undulatory}, with the formulation of Density Functional Theory (DFT) \supercite{thomas1927calculation,fermi1927statistical, hogenberg, kohn}, a theory based on the electron density rather than many-body wavefunctions reducing the dimensional scaling from exponential to cubic.   

	Even though DFT improves the dimensional scaling, the computational scaling is steep concerning the system size \supercite{RN47} representing a limitation in the studying of bulk systems as liquids, i.e., an infinite number of molecules. To ameliorate DFT to be applied in bulk systems, a divide-and-conquer strategy emerged, in which the total electron density is divided into a sum of individual densities, which effectively scales almost linearly concerning the system size, called subsystem DFT \supercite{mi2021eqe,mi2019nonlocal,mi2019ab}.

	By partitioning a bulk-phase system into smaller subsystems allows the selection of a few or even one subsystem of interest, which as a consequence is embedded in a complex environment \supercite{schmitt2020frozen}. This arises to a robust strategy to study the effects of the environment due to a single subsystem, the so-called “Embedding Potential” \supercite{genova2016avoiding}. Furthermore, making use of the embedding potential, a new subsystem DFT method capable of approaching bulk charged system regardless of the use of periodic boundary conditions (PBC) was presented as Impurity Model \supercite{tolle2019charged}. 

	In this work, the so-called “Impurity Model” and the use of the $\Delta$-SCF approach (the energy difference between the neutral and the polarized systems) are paired together to determine the vertical IP of some molecular condensed phase systems, as liquids and water. 
	
	This statement is divided into four main sections. A theoretical background with the basis of DFT and subsystem DFT, $\Delta$-SCF method, and the Impurity model. A second section regarding the computational details and the systems of interest. The third part contains the results and discussion, and the last close with the state of the art of the current project and future directions. 


\section{Theoretical Background}

	\subsection{DFT and KS-DFT}

	\supercite{ullrich2012time}
	\supercite{parr1980density}	
	\supercite{wesolowski2007hohenberg}

	\subsection{Subsystem DFT}
	Subsystem DFT or Frozen Density embedding formalism developed by Wesolowski and Warshel employs a divide-and-conquer approach where the Kohn--Sham (KS) problem is solved for fragments of small size, the so-called subsystems, which alleviate the $n^3$ scaling of KS-DFT\supercite{sDFT,wesolowski1993frozen,wesolowski2006one}. In subsystem-DFT the electron density of the total system $\rho$ is partitioned into subsystem electron densities and can be written as

	\begin{equation}\label{eq:sumdensity}
		\rho(r) = \sum\limits_{I=1}^{ \# \ of \ subsystems} \rho_{I}(r),
	\end{equation}

	where the index $I$ runs over all the subsystems.
	
	The subsystem densities $\rho_I$ are constructed from the absolute square of all subsystem-specific orbitals $\psi_{i}^I(r)$ as

	\begin{equation}\label{eq:subso}
		\rho_I( r) = \sum\limits_{i=1}^{n_I} \abs{\psi_{i}^I(r)}^2,
	\end{equation}
	
	where $n_I$ accounts for the total number of electrons per subsystem $I$.
	The total electronic energy becomes a function of all the subsystem densities and has the form,

	\begin{equation}
		E[\{\rho_I\}] = \sum\limits_I T_{s}[\rho_I] + V_{\mathrm{nuc}}[\rho] + E_H[\rho] + E_{\mathrm{xc}}[\rho] + T_{s}^{\mathrm{nad}}[\{\rho_I\}],
	\end{equation}
	
	where $V_{\mathrm{nuc}}[\rho]=\int \rho(r) v_\mathrm{ext}(r)dr$, and $E_H[\rho]$ and $E_{\mathrm{xc}}[\rho]$ are the Hartree  and exchange-correlation energy functionals.
	
	Conventionally, the sum of the single subsystem kinetic energy contributions $T_s[\rho_I]$ is not equal to the total non-interacting kinetic energy of the system. In such a way, a non-additive contribution arises, which is captured by the non-additive kinetic energy functional (NAKE), defined as

	\begin{equation}
		\label{nad}
		T_{s}^{\mathrm{nad}}[\{\rho_{I}\}] = T_{s}[\rho] - \sum\limits_{I=1}^{N_S} T_{s}[\rho_{I}].
	\end{equation}

	The variational problem is cleared up by solving self-consistently a KS like equation per subsystem with constrained electron density \supercite{wesolowski1994ab}, which reads

	\begin{equation}\label{eq:ksced}
		\left( -\frac{\nabla^2}{2} + v_{\mathrm{eff}}^{I}[\rho_I](r) + v_{\mathrm{emb}}^{I}[\rho_{I},\rho](r) \right) \psi_{i}^I(r) = \epsilon_{i}^I \psi_{i}^I(r)
	\end{equation}
	The subsystem orbital energies are denoted by $\epsilon_{i}^I$ in Eq.\ \ref{eq:ksced}.\supercite{sDFT}, and the subsystem effective KS potential ($v_{\mathrm{eff}}^{I}[\rho_I](r)$) is defined as

	\begin{equation}\label{equ6}
		v_{\mathrm{eff}}^{I}[\rho_I](r) = v_{\mathrm{ext}}^I(r) + v_{{H}}[\rho_I](r) + v_{\mathrm{xc}}[\rho_I](r).
	\end{equation}

	In equation \ref{equ6},  $v_{\mathrm{ext}}^I(r)$ is the  external potential of subsystem $I$ while $v_{\mathrm{H}}[\rho_I](r)$ and $v_{\mathrm{xc}}[\rho_I](r)$ represent the Hartree and the exchange-correlation potentials per subsystem $I$.

	By way of explanation, the Kohn–Sham single-particle Hamiltonian is augmented by an embedding potential, in which  are encoded the interaction of the electrons of subsystem $I$ with the environment, which reads as follows,

	\begin{equation}
		\begin{aligned}
			v_{\mathrm{emb}}^{I}[\rho,\rho_I](r) = \sum\limits_{J,J \neq I} &v_{\mathrm{ext}}^{\; J}(r) + v_{{H}}[\rho - \rho_I](r)  \\ + \ &v_{\mathrm{xc}}^{\mathrm{nad}}[\rho,\rho_{I}](r) + v_{T_s}^{\mathrm{nad}}[\rho,\rho_{I}](r),
		\end{aligned}
	\end{equation}
	where $v_{\mathrm{ext}}^{\; J}$ and $v_{{H}}[\rho - \rho_I](r)$ are the inter-subsystem potentials that capture the Coulomb interactions. In detail, the non-additive exchange-correlation and kinetic potential of the environment $v_{\mathrm{xc}}^{\mathrm{nad}}[\rho,\rho_{I}](r)$ is defined as
	
	\begin{equation}
		\label{emb_xct}
		v_{\mathrm{xc, T_{S}}}^{\mathrm{nad}}[\rho,\rho_{I}](r) = \frac{\delta E_{\mathrm{xc, T_{S}}}[\rho]}{\delta \rho(r)} - \frac{\delta E_{\mathrm{xc, T_{S}}}[\rho_I]}{\delta \rho_I(r)}.
	\end{equation}

	In practice, this set of equations can either be solved iteratively using  a Freeze-and-Thaw\supercite{FaT} procedure(as done in ADF\supercite{te2001chemistry} and other molecular codes), or simultaneously by all subsystems where the total density is updated at every SCF cycle (as done in CP2K \supercite{hutter2014cp2k} and Quantum-Espresso\supercite{RN48,mi2021eqe}).
	
\subsection{$\Delta$SCF procedure}

	$\Delta$SCF procedure is used to compute ionization energies\supercite{krishtal2015subsystem}. Here, for the ionization potential this method involves taking the difference between the energy of the neutral system, $E_{\mathrm{tot}}^\circ$, and the energy of the charged system, $E_{\mathrm{tot}}^+$, i.e.,

	\begin{equation}
		\label{IP}
		IP = E_{\mathrm{tot}}^+ - E_{\mathrm{tot}}^\circ.
	\end{equation}
	
	In practical calculations, the bulk system is represented in periodic boundary conditions (PBCs) with simulation cells containing a small number of molecules.The IP of liquid water is therefore sampled over the total number of molecules per simulation cell sampling over several snapshot configurations. 
	
	We thus have an opportunity and a potential problem. On one side, we can use subsystem DFT to compute the electronic structure of ionized systems avoiding spin-density delocalization issues arising from the self-interaction in the exchange-correlation density functional \supercite{ciofini2005self,bao2018self} as well as with a quasi-linear scaling cost. On the other side, because of the need of PBCs, the total energy of charged systems is not directly accessible\supercite{duan2000ewald}. Later we outline a workaround to the problem which we call ``impurity model''. 

\subsection{The Impurity Model}\label{IM}
	The impurity model relies on the fact that a subsystem electron density can be taken for the density of an extended, periodic subsystem or the density of a finite, single subsystem. In practical calculations and under the condition that the exchange-correlation functional approximation is local or semilocal, the difference between the finite and the extended subsystems is localized, in such a way that the Coulomb potentials for the electronic and nuclear charges are computed. 

	For a neutral finite subsystem, the Hartree embedding potential is augmented by the so-called screening potential to become $v_H[\rho-\rho_I](r)+v^{\mathrm{screen}}_H[\rho_I](r)$, with 
	\begin{equation}
		\label{sc1}
		v^{\mathrm{screen}}_H[\rho_I](r) = v_H[\rho_I](r) - \bar{v}_H[\rho_I](r),
	\end{equation}
	where $\bar{v}_H$ is the Hartree potential of a finite, nonperiodic subsystem with density $\rho_I$. 
	
	The screening potential arises for the contribution of the embedding potential of the periodic images regarding the same subsystem, generating a partition of the system into a finite, nonperiodic subsystem with density $\rho_I$ and a semi-infinite environment. This impurity model is then applied to the nuclear coulomb potential as well.
	
	The screening potential for a charged subsystem requires more attention because of the need to keep the charged subsystem in one lattice site and replace the charged subsystem with neutral subsystems in all the other periodic images to avoid the decay of the Coulomb potential. The above is solved by considering the following screening potential, which depend on the subsystem electron densities of the charged, $\rho_I^\circ$, and neutral, $\rho_I^+$, subsystems. Namely,
	
	\begin{equation}
		\label{sc2}
		v^{\mathrm{screen}}_H[\rho_I^\circ,\rho_I^+](r) = v_H[\rho_I^\circ](r) - \bar{v}_H[\rho_I^\circ](r) + \Delta_0,
	\end{equation}
	
	where $\Delta_0$ is a constant shift added to the embedding potential to reference neutral and charged systems (i.e., the so-called $G=0$ correction\supercite{duan2000ewald}). Here, an auxiliary neutral subsystem density $\rho_I^\circ$ is taken from a separate neutral system calculation. 
	
	Once the embedding potential for both neutral and charged systems are obtained, another common workflow is to approximate the energy of subsystem-environment interaction (denoted by $E_{\mathrm{int},I}[\rho_{I},\rho-\rho_{I}]$) by an integral over the embedding potential. Namely, 
	%
	\begin{equation}\label{emb_int}
		E_{\mathrm{int},I}[\rho_{I},\rho-\rho_{I}] \simeq \int v^I_{\mathrm{emb}}[\rho, \rho_I](r)  \rho_{I}(r) \mathrm{d} r.
	\end{equation}
	
	Such an approximation was used previously, recovering good results \supercite{tolle2019charged}. However, a relaxed approximation of Eq.\ \ref{emb_int} can be made by simply evaluating the total nonadditive functionals computed similarly to Eq.\ \ref{nad} for NAKE, exchange-correlation, and Coulomb potentials, which target the quantitative determination of the ionization potential in bulk systems.

\section{Computational Details}
	Summarize of computational details of the paper
\section{Results}
	Summarize of paper results
	
	Successfully, we applied this method to the current state-of-the-art liquid water IPs, reproducing the experimental values to within 0.5 eV. They were determined averaging over two different systems of 64 and 256 water molecules (or subsystems) considered in the corresponding simulation cell. The technique provides an exhaustive breakdown of the contributions to the IP of liquid water: mean-field, correlation, embedding, and environment polarization.   
\section{Current project and future directions}
	Our future work will show that density embedding is convenient for computing observables challenging to access by standard methods in other liquids systems different from water. And shed light on the main physical effects underpinning the modeled quantity beyond the capabilities of standard DFT.

	The schedule of this project will cover three years, including the liquids subsystem DFT quantum molecular dynamics simulations.  The calculation of Embedding potentials for ionized and nonionized systems with embedded Quantum ESPRESSO (eQE) and the new self-develop python-based code: eDFTpy and QEPY.

\section{Conclusions}


\printbibliography




\end{document}
